\documentclass[a4paper,11pt]{article}
\usepackage[utf8]{inputenc}
\usepackage[usenames,dvipsnames]{color}
\usepackage{graphicx}
\usepackage{fullpage}
\usepackage[justification=centering,labelfont=bf]{caption}
\usepackage{listings}
\usepackage[hidelinks]{hyperref}

\makeatletter
\g@addto@macro\@floatboxreset\centering
\makeatother

\newcommand{\answerspace}{\vspace{0.5cm}}
\newcommand{\figurespace}{\vspace{0.6cm}}

\begin{document}
\begin{titlepage}
\begin{center}
\textsc{\Large Parallelism}
\\
\texttt{1202}
\\[1.5cm]
\rule{\linewidth}{0.5mm}
\\[0.4cm]
{\huge
\bfseries
Lab 1: Embracing parallelism with OpenMP: Mandelbrot set
\\[0.4cm]
}
\rule{\linewidth}{0.5mm}
\\[2.5cm]
\begin{minipage}{0.4\textwidth}
\begin{flushleft}
\large
Héctor Ramón Jiménez
\end{flushleft}
\end{minipage}
\begin{minipage}{0.4\textwidth}
\begin{flushright}
\large
Alvaro Espuña Buxo
\end{flushright}
\end{minipage}
\vfill
{\large
\today
}
\\
{\large
\texttt{Facultat d'Informàtica de Barcelona}
}
\end{center}
\end{titlepage}
\section{Reading activity}
In this first part of the report we first briefly describe the basic formulation for the \textbf{Mandelbrot set}
    (see \ref{mandelbrot:references}) and then its implementation in the sequential code available (\texttt{mandel-serial.c}).
\subsection{Description}
The \textbf{Mandelbrot set} is a particular set of points in the complex domain named after the mathematician
    \textbf{Benoit Mandelbrot}, who studied it and popularized it. The \textbf{Mandelbrot set}'s boundary is an easily
    recognizable two-dimensional fractal shape (see figure \ref{mandelbrot:boundary}).
\begin{figure}[h!]
\includegraphics[width=0.70\textwidth]{figures/mandelbrot_boundary.jpg}
\caption{The Mandelbrot set boundary}
\label{mandelbrot:boundary}
\end{figure}
\\
Mandelbrot set images are made by sampling complex numbers and determining for each whether the result
    \textbf{tends towards infinity} when a particular mathematical operation is \textbf{iterated} on it. The real
    and imaginary parts of each number are treated as image coordinates. The pixels are colored according to how rapidly
    the sequence diverges.
\\
More precisely, the \textbf{Mandelbrot set} is the set of values of $c$ in the complex plane for which the orbit of
    $0$ under iteration of the complex quadratic polynomial $z_{n+1} = z^2_n + c$ remains bounded.
\subsection{Algorithm}
The simplest algorithm for drawing a picture of the Mandelbrot set is the following. We \textbf{discretize} the
    complex space in a set of points and each point corresponds to a \textbf{pixel} in a two dimensional plot. To color
    any such pixel, let $c$ (represented by the complex variable $c$ at the \texttt{mandel-serial.c}
    code) be the midpoint of that pixel. Then we calculate $z_0$, $z_1$, ... (stored in the complex variable
    $z$) and beyond until \textbf{divergence} occurs or
    the \textbf{maximum number of iterations} is reached. We assume that divergence happens when the resulting complex
    $z_j$ is \textbf{not contained in the problem space}. More precisely, let the problem space be
    $\{ (r, i) \: | -\!N < r < N, \: -N < i < N \}$, then divergence occurs when:
\[
length(z_j) \geq N
\]
\\
The \textbf{intensity of the color} of the point $c$ is directly proportional to the \textbf{number of iterations performed}
    $k_c$ without divergence. Thus the points that belong to the \textbf{Mandelbrot set} are going to be the most intense ones
    (usually black color). That can be easily done calculating the \texttt{scale\_color} variable, which is the factor that needs
    to be applied to the \texttt{min\_color} for every performed iteration:
\[
\texttt{scale\_color}
=
\frac{\texttt{max\_color} - \texttt{min\_color}}{\texttt{maxiter} - 1}
\]
\[
\texttt{color}
=
(k_c - 1) \cdot \texttt{scale\_color} + \texttt{min\_color}
\]
\subsection{References}
\label{mandelbrot:references}
\begin{enumerate}
\item{Mandelbrot set - Wikipedia, the free encyclopedia. [Internet]. url:
    \url{http://en.wikipedia.org/wiki/Mandelbrot\_set} (visited on Apr 23, 2014).}
\end{enumerate}
\clearpage
\section{Task granularity analysis}
\begin{enumerate}
\setcounter{enumi}{0}
\item
\textbf{Which are the two most important common
    characteristics of the task graphs generated for the two task
    granularities (\emph{Row} and \emph{Point}) for
    \textbf{mandel-tareador}?  Include the task graphs that are
    generated in both cases for \texttt{-w 8}.}
\\
There are no data dependencies between tasks.
\begin{figure}[h!]
\includegraphics[width=1.0\textwidth]{figures/point_deps_mandel.pdf}
\caption{Task decomposition with \emph{Point} granularity.}
\label{figure:mandel-point}
\end{figure}
\begin{figure}[h!]
\includegraphics[width=1.0\textwidth]{figures/row_deps_mandel.pdf}
\caption{Task decomposition with \emph{Row} granularity.}
\label{figure:mandel-row}
\end{figure}
\setcounter{enumi}{1}
\item
\textbf{Which section of the code is causing the serialization of
    all tasks in mandeld-tareador? Include the task graph generated for
    the graphical version of Point after isolating the section of the
    code}
\newpage
\setcounter{enumi}{2}
\item
\textbf{Using the results obtained from the simulated parallel
    execution for mandel-tareador and for a size of -w 16, complete the
    following table with the execution time and speed-up (with respect to
    the execution with 1 processor) obtained for the non-graphical
    version, for both individual task granularities. Comment the results
    highlighting the reason for the poor scalability, if detected?}

\answerspace
  The results obtained of simulating the execution of
  \texttt{mandel-tareador} with \emph{Paraver} are:

  \figurespace
  \begin{center}
    \begin{tabular}{| c || c | c | c | c |}
      \hline
      & \multicolumn{2}{| c |}{\bf Row} & \multicolumn{2}{| c |}{\bf Point} \\
      \hline
      \textbf{Num. processors} & \textbf{Time (ms)} & \textbf{Speed-up} & \textbf{Time (ms)} & \textbf{Speed-up} \\
      \hline\hline
      \textbf{1} & 1080.584 & 1.000 & 1120.216 &  1.000 \\ \hline
      \textbf{2} &  554.698 & 1.948 &  563.475 &  1.988 \\ \hline
      \textbf{4} &  334.376 & 3.231 &  306.504 &  3.654 \\ \hline
      \textbf{8} &  330.156 & 3.272 &  163.536 &  6.849 \\ \hline
      \textbf{16} &  329.724 & 3.277 &   93.059 & 12.037 \\ \hline
    \end{tabular}
  \end{center}
  \figurespace
  We can see how \emph{Point} granularity seems to scale better than
  \emph{Row} granularity. At first, with two and four processors, we
  can see that both execution times are reduced and scale reasonably
  well, but as we add more processors, \emph{Row} stalls.

  The problem is that, with \emph{Row} granularity we have 16 tasks, and
  as we saw earlier, there are tasks that take a lot of time. We have a load balance
  problem, since some processor will have to execute the longest task while
  other processors are done and waiting. The longest row becomes the bottleneck,
  and adding more processors, won't produce speed-up.

  \figurespace
\begin{figure}[h!]
\includegraphics[width=0.7\textwidth]{figures/row_16_cores.png}
\caption{Row granularity: Most of the threads are in idle waiting for the longest task to finish.}
\label{figure:load-row}
\end{figure}
  \figurespace

  With \emph{Point} granularity, since we have 256 tasks ($16 \cdot 16$) while
  the longest points are being computed, the other processors can do other tasks,
  hence longest points are not the bottleneck now.

  \figurespace
\begin{figure}[h!]
\includegraphics[width=0.7\textwidth]{figures/point_16_cores.png}
\caption{Point granularity: Since there are many more tasks than threads, the load can be distributed better.}
\label{figure:load-point}
\end{figure}
  \figurespace
  From this exercise we can conclude that when $n_{processors} <<
  n_{tasks}$ is more difficult to face load balance problems.

\end{enumerate}
\clearpage
\section{OpenMP execution analysis}
\begin{enumerate}
\setcounter{enumi}{3}
\item
\textbf{For each parallelization strategy of the non-graphical version, complete the following table with
    the execution time for different loop schedules and number of threads, reasoning about the results
    that are obtained.}
\begin{center}
\begin{tabular}{| c || c | c | c | c |}
\hline
\textbf{Num. threads} & \textbf{static} & \textbf{static, 1} & \textbf{dynamic, 1} & \textbf{guided, 1}
\\
\hline
\hline
1 & 30.207 & 30.207 & 30.207 & 30.207
\\
\hline
2 & 15.223 & 15.233 & 15.206 & 15.224
\\
\hline
4 & 15.228 & 7.715 & 7.612 & 16.654
\\
\hline
6 & 14.716 & 5.391 & 5.334 & 11.438
\\
\hline
8 & 12.757 & 4.025 & 4.025 & 8.147
\\
\hline
10 & 10.924 & 3.364 & 3.376 & 7.127
\\
\hline
12 & 9.580 & 2.802 & 2.817 & 6.006
\\
\hline
\end{tabular}
\end{center}
[...]
\setcounter{enumi}{4}
\item
\textbf{For each parallelization strategy, complete the following table with the information extracted from
    the Extrae instrumented executions with 8 threads and analysis with Paraver, reasoning about the
    results that are obtained.}
\begin{center}
\begin{tabular}{| c || c | c | c | c |}
\hline
\textbf{} & \textbf{static} & \textbf{static, 1} & \textbf{dynamic, 1} & \textbf{guided, 1}
\\
\hline
\hline
Running avg. time per thread (s) & 3.800 & 3.815 & 3.804
\\
\hline
Execution unbalance (s) & 0.298 & 0.298 & 0.300
\\
\hline
SchedForkJoin ($\mu s$) & 85.227 & 1.599 & 1.584 &
\\
\hline
\end{tabular}
\end{center}
[...]
\end{enumerate}
\end{document}
