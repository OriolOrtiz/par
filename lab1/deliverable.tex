\documentclass[a4paper,11pt]{article}
\usepackage[utf8]{inputenc}
\usepackage[usenames,dvipsnames]{color}
\usepackage{graphicx}
\usepackage[justification=centering,labelfont=bf]{caption}
\usepackage{listings}
\begin{document}
\begin{titlepage}
\begin{center}
\textsc{\Large Parallelism}
\\
\texttt{1202}
\\[1.5cm]
\rule{\linewidth}{0.5mm}
\\[0.4cm]
{\huge
\bfseries
Lab 1: Embarrassingly parallelism with OpenMP: Mandelbrot set
\\[0.4cm]
}
\rule{\linewidth}{0.5mm}
\\[2.5cm]
\begin{minipage}{0.4\textwidth}
\begin{flushleft}
\large
Héctor Ramón Jiménez
\end{flushleft}
\end{minipage}
\begin{minipage}{0.4\textwidth}
\begin{flushright}
\large
Alvaro Espuña Buxo
\end{flushright}
\end{minipage}
\vfill
{\large
\today
}
\\
{\large
\texttt{Facultat d'Informàtica de Barcelona}
}
\end{center}
\end{titlepage}
\section{Reading activity}
In this first part of the report we first briefly describe the basic formulation for the Mandelbrot set [...]
    [Here you should add an identification of the reference that you have used to write the simple description
    of the problem, and that you should include below.] and then its implementation in the sequential code
    available (mandel-serial.c). The Mandelbrot set is a particular set of points, in the complex domain ...
    [Here you should write about the complex space, the recurrence polynomial that is applied to each point
    in that complex space and the condition used to decide if it belongs or not to the set.]
    The simplest algorithm for drawing a picture of the Mandelbrot set is the following. We discretize the
    complex space in a set of points and each point corresponds to a pixel in a two dimensional plot. To color
    any such pixel, let c (represented by the complex variable ...... [Complete it] at the mandel-serial.c
    code) be the midpoint of that pixel. ... [ Here you should continue the description of the sequential
    algorithm clearly naming the variables that are used in mandel-serial.c and the criteria used to decide
    the color for each point.]
    [...] ... [Here you should insert the identification reference and the description of that reference: title,
    authors, year of publication, website, etc.]
\end{document}
