\documentclass[a4paper,11pt]{article}
\usepackage[utf8]{inputenc}
\usepackage[usenames,dvipsnames]{color}
\usepackage{graphicx}
\usepackage[justification=centering,labelfont=bf]{caption}
\usepackage{listings}
\usepackage[hidelinks]{hyperref}
\begin{document}
\begin{titlepage}
\begin{center}
\textsc{\Large Parallelism}
\\
\texttt{1202}
\\[1.5cm]
\rule{\linewidth}{0.5mm}
\\[0.4cm]
{\huge
\bfseries
Lab 1: Embarrassingly parallelism with OpenMP: Mandelbrot set
\\[0.4cm]
}
\rule{\linewidth}{0.5mm}
\\[2.5cm]
\begin{minipage}{0.4\textwidth}
\begin{flushleft}
\large
Héctor Ramón Jiménez
\end{flushleft}
\end{minipage}
\begin{minipage}{0.4\textwidth}
\begin{flushright}
\large
Alvaro Espuña Buxo
\end{flushright}
\end{minipage}
\vfill
{\large
\today
}
\\
{\large
\texttt{Facultat d'Informàtica de Barcelona}
}
\end{center}
\end{titlepage}
\section{Reading activity}
In this first part of the report we first briefly describe the basic formulation for the \textbf{Mandelbrot set}
    (see \ref{mandelbrot:references}) and then its implementation in the sequential code available (\texttt{mandel-serial.c}).
\subsection{Description}
The \textbf{Mandelbrot set} is a particular set of points in the complex domain named after the mathematician
    \textbf{Benoit Mandelbrot}, who studied it and popularized it. The \textbf{Mandelbrot set}'s boundary is an easily
    recognizable two-dimensional fractal shape (see figure \ref{mandelbrot:boundary}).
\begin{figure}[h!]
\includegraphics[width=1.0\textwidth]{figures/mandelbrot_boundary.jpg}
\caption{The Mandelbrot set boundary}
\label{mandelbrot:boundary}
\end{figure}
\\
Mandelbrot set images are made by sampling complex numbers and determining for each whether the result
    \textbf{tends towards infinity} when a particular mathematical operation is \textbf{iterated} on it. The real
    and imaginary parts of each number are treated as image coordinates. The pixels are colored according to how rapidly
    the sequence diverges.
\\
More precisely, the \textbf{Mandelbrot set} is the set of values of $c$ in the complex plane for which the orbit of
    $0$ under iteration of the complex quadratic polynomial $z_{n+1} = z^2_n + c$ remains bounded.
\subsection{Algorithm}
The simplest algorithm for drawing a picture of the Mandelbrot set is the following. We \textbf{discretize} the
    complex space in a set of points and each point corresponds to a \textbf{pixel} in a two dimensional plot. To color
    any such pixel, let $c$ (represented by the complex variable $c$ at the \texttt{mandel-serial.c}
    code) be the midpoint of that pixel. Then we calculate $z_0$, $z_1$, ... (stored in the complex variable
    $z$) and beyond until \textbf{divergence} occurs or
    the \textbf{maximum number of iterations} is reached. We assume that divergence happens when the resulting complex
    $z_j$ is \textbf{not contained in the problem space}. More precisely, let the problem space be
    $\{ (r, i) \: | -\!N < r < N, \: -N < i < N \}$, then divergence occurs when:
\[
length(z_j) \geq N
\]
\\
The \textbf{intensity of the color} of the point $c$ is directly proportional to the \textbf{number of iterations performed}
    $k_c$ without divergence. Thus the points that belong to the \textbf{Mandelbrot set} are going to be the most intense ones
    (usually black color). That can be easily done calculating the \texttt{scale\_color} variable, which is the factor that needs
    to be applied to the \texttt{min\_color} for every performed iteration:
\[
\texttt{scale\_color}
=
\frac{\texttt{max\_color} - \texttt{min\_color}}{\texttt{maxiter} - 1}
\]
\[
\texttt{color}
=
(k_c - 1) \cdot \texttt{scale\_color} + \texttt{min\_color}
\]
\subsection{References}
\label{mandelbrot:references}
\begin{enumerate}
\item{Mandelbrot set - Wikipedia, the free encyclopedia. [Internet]. url:
    \url{http://en.wikipedia.org/wiki/Mandelbrot\_set} (visited on Apr 23, 2014).}
\end{enumerate}
\clearpage
\section{Task granularity analysis}
[...]
\clearpage
\section{OpenMP execution analysis}
\begin{enumerate}
\setcounter{enumi}{0}
\item
\textbf{For each parallelization strategy of the non-graphical version, complete the following table with
    the execution time for different loop schedules and number of threads, reasoning about the results
    that are obtained.}
\begin{center}
\begin{tabular}{| c || c | c | c | c |}
\hline
\textbf{Num. threads} & \textbf{static} & \textbf{static, 1} & \textbf{dynamic, 1} & \textbf{guided, 1}
\\
\hline
\hline
1 & 30.207 & 30.207 & 30.207 & 30.207
\\
\hline
2 & 15.223 & 15.233 & 15.206 & 15.224
\\
\hline
4 & 15.228 & 7.715 & 7.612 & 16.654
\\
\hline
6 & 14.716 & 5.391 & 5.334 & 11.438
\\
\hline
8 & 12.757 & 4.025 & 4.025 & 8.147
\\
\hline
10 & 10.924 & 3.364 & 3.376 & 7.127
\\
\hline
12 & 9.580 & 2.802 & 2.817 & 6.006
\\
\hline
\end{tabular}
\end{center}
[...]
\setcounter{enumi}{1}
\item
\textbf{For each parallelization strategy, complete the following table with the information extracted from
    the Extrae instrumented executions with 8 threads and analysis with Paraver, reasoning about the
    results that are obtained.}
\begin{center}
\begin{tabular}{| c || c | c | c | c |}
\hline
\textbf{} & \textbf{static} & \textbf{static, 1} & \textbf{dynamic, 1} & \textbf{guided, 1}
\\
\hline
\hline
Running avg. time per thread &  &  &  & 
\\
\hline
Execution unbalance &  &  &  & 
\\
\hline
SchedForkJoin &  &  &  & 
\\
\hline
\end{tabular}
\end{center}
[...]
\end{enumerate}
\end{document}
